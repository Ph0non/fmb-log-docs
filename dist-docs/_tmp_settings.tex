% Options for packages loaded elsewhere
\PassOptionsToPackage{unicode}{hyperref}
\PassOptionsToPackage{hyphens}{url}
\documentclass[
]{article}
\usepackage{xcolor}
\usepackage{amsmath,amssymb}
\setcounter{secnumdepth}{-\maxdimen} % remove section numbering
\usepackage{iftex}
\ifPDFTeX
  \usepackage[T1]{fontenc}
  \usepackage[utf8]{inputenc}
  \usepackage{textcomp} % provide euro and other symbols
\else % if luatex or xetex
  \usepackage{unicode-math} % this also loads fontspec
  \defaultfontfeatures{Scale=MatchLowercase}
  \defaultfontfeatures[\rmfamily]{Ligatures=TeX,Scale=1}
\fi
\usepackage{lmodern}
\ifPDFTeX\else
  % xetex/luatex font selection
\fi
% Use upquote if available, for straight quotes in verbatim environments
\IfFileExists{upquote.sty}{\usepackage{upquote}}{}
\IfFileExists{microtype.sty}{% use microtype if available
  \usepackage[]{microtype}
  \UseMicrotypeSet[protrusion]{basicmath} % disable protrusion for tt fonts
}{}
\makeatletter
\@ifundefined{KOMAClassName}{% if non-KOMA class
  \IfFileExists{parskip.sty}{%
    \usepackage{parskip}
  }{% else
    \setlength{\parindent}{0pt}
    \setlength{\parskip}{6pt plus 2pt minus 1pt}}
}{% if KOMA class
  \KOMAoptions{parskip=half}}
\makeatother
\setlength{\emergencystretch}{3em} % prevent overfull lines
\providecommand{\tightlist}{%
  \setlength{\itemsep}{0pt}\setlength{\parskip}{0pt}}
% PDF header for Pandoc output
\usepackage[most]{tcolorbox}
\usepackage{xcolor}
\usepackage{siunitx}

% Prefer upright/consistent units in German docs
\sisetup{
  detect-all,
  per-mode=symbol
}

% Corporate colors (see src/styles/theme.css)
\definecolor{FMBBlue}{HTML}{217CA3}
\definecolor{FMBGreen}{HTML}{63A90F}
\definecolor{FMBAnthrazit}{HTML}{3D3E3D}
\definecolor{FMBOrange}{HTML}{F17119}
\definecolor{FMBBrombeer}{HTML}{A70050}
\definecolor{FMBPetrol}{HTML}{006E78}
\definecolor{FMBGrey}{HTML}{888888}

% Consistent spacing for admonition boxes (tcolorbox)
\tcbset{
  before skip=8pt,
  after skip=8pt
}
\usepackage{bookmark}
\IfFileExists{xurl.sty}{\usepackage{xurl}}{} % add URL line breaks if available
\urlstyle{same}
\hypersetup{
  hidelinks,
  pdfcreator={LaTeX via pandoc}}

\author{}
\date{}

\begin{document}

\section{Einstellungen}\label{einstellungen}

\subsection{Datenbankpfad}\label{datenbankpfad}

Standard ist eine lokale SQLite‑Datei im Programmordner unter
\texttt{resources}. Dieser Ort ist bewusst gewählt: Die Anwendung kann
dort mit definierten Pfaden arbeiten und bei Updates eine vorhandene
Datenbank weiterverwenden, ohne sie zu überschreiben.

Optional kann eine andere DB-Datei ausgewählt werden (z.\,B.
Netzlaufwerk).

Ein Netzlaufwerk kann sinnvoll sein, wenn mehrere Arbeitsplätze auf
dieselbe Datenbank zugreifen sollen. Ob das zuverlässig funktioniert,
hängt jedoch stark von Dateisperren und der SMB‑Konfiguration ab.

\begin{tcolorbox}[enhanced,breakable,boxrule=0.6pt,arc=1mm,left=1.5mm,right=1.5mm,top=1mm,bottom=1mm,colframe=FMBOrange,colback=FMBOrange!10,fonttitle=\bfseries,title={Hinweis (Mehrbenutzerbetrieb)}]

\begin{itemize}
\tightlist
\item
  SQLite ist serverlos und nutzt Datei‑Locking. Auf manchen Shares kann
  das zu sporadischen Sperren führen.
\item
  Testen Sie den Mehrbenutzerbetrieb unter realistischen Bedingungen und
  planen Sie Backups ein.
\end{itemize}

\end{tcolorbox}

\begin{tcolorbox}[enhanced,breakable,boxrule=0.6pt,arc=1mm,left=1.5mm,right=1.5mm,top=1mm,bottom=1mm,colframe=FMBBlue,colback=FMBBlue!6,fonttitle=\bfseries,title={Kurzfassung}]

\begin{itemize}
\tightlist
\item
  Standard: \texttt{resources} im Programmordner
\item
  Optional: anderer Pfad (z.\,B. Netzlaufwerk)
\item
  Stabilität bei Mehrbenutzerbetrieb hängt von SMB/Locking ab
\end{itemize}

\end{tcolorbox}

\subsection{Protokoll‑Cache (lokal)}\label{protokollcache-lokal}

Damit Messprotokolle auch ohne Hub‑Verbindung angezeigt werden können,
nutzt FMB Log einen \textbf{lokalen Protokoll‑Cache} im Benutzerprofil.
Beim Import werden Protokolle immer in diesen Cache geschrieben. Ist der
Hub erreichbar, werden sie zusätzlich in das Protokoll‑Archiv im
Hub‑Ordner übertragen; ist der Hub nicht erreichbar, wird der Upload
später automatisch nachgeholt.

Sie können den Cache in den Einstellungen anzeigen lassen und bei Bedarf
leeren.

\begin{tcolorbox}[enhanced,breakable,boxrule=0.6pt,arc=1mm,left=1.5mm,right=1.5mm,top=1mm,bottom=1mm,colframe=FMBOrange,colback=FMBOrange!10,fonttitle=\bfseries,title={Hinweis (Cache leeren)}]

\begin{itemize}
\tightlist
\item
  Wenn Sie den Cache leeren, können Protokolle offline nicht mehr
  angezeigt werden, bis der Hub wieder erreichbar ist.
\item
  Leeren Sie den Cache nur, wenn der Hub erreichbar ist und ausstehende
  Uploads (Offline‑Import) bereits nachgetragen wurden, da sonst das
  spätere Hochladen nicht mehr möglich ist.
\end{itemize}

\end{tcolorbox}

\end{document}
